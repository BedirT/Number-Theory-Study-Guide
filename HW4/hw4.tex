\documentclass[12pt]{article}
%\documentstyle[12pt]{article}
\usepackage{amsmath, amssymb, listings, lmodern, xcolor, titlesec, booktabs, geometry}
\usepackage{enumitem}

%\usepackage{proof}

\geometry{
	margin = 0.5in
}

\lstset{
	basicstyle=\ttfamily,
	columns=fullflexible,
	frame=single,
	breaklines=true,
	postbreak=\mbox{\textcolor{red}{$\hookrightarrow$}\space},
}

\renewcommand{\phi}{\varphi}
\newcommand{\logequiv}{{\models =\!\!\!|}}

\setlength\parindent{0pt}
%\newcommand{\sectionbreak}{\clearpage}

\title{Number Theopry HW4 - Part 1}

\author{M Bedir Tapkan}

\begin{document}
	\maketitle\textsl{}

\section*{Solution 1}



\section*{Solution 2}

\section*{Solution 11}
 \textbf{a:}
 \\\\
 \textbf{b:} 
 $q | 3^p + 1 \implies 3^p \equiv -1 (mod q) \implies 3^{2p} \equiv 1 (mod q)$\\

\section*{Solution 12}
From Fermat's Little Theorem we know that $a^{p-1} \equiv 1 (mod p)$. In our case it will be $3^{100} \equiv 1 (mod 101)$ since $101$ is a prime number.\\
We also know that:\\
$3^{32,123,878,237,982,731,602}  = 3^2 \times 3^{32,123,878,237,982,731,600} = 3^2 \times (3^{100})^{321,238,782,379,827,316}$\\
$(3^{100})^{321,238,782,379,827,316} \equiv 3^2 \times 1^{321,238,782,379,827,316} (mod 101) \equiv 9 (mod 101)$

\section*{Solution 13}
\textbf{a:} $1234 = 2^{10} + 2^7 + 2^6 + 2^4 + 2^1$\\

\end{document}